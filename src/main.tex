\documentclass[11p]{article}
% Packages
\usepackage{amsmath}
\usepackage{graphicx}
\usepackage[swedish]{babel}
\usepackage[
    backend=biber,
    style=authoryear-ibid,
    sorting=ynt
]{biblatex}
\usepackage[utf8]{inputenc}
\usepackage[T1]{fontenc}
%Källor
\addbibresource{mall.bib}
\graphicspath{ {./images/} }

\title{Kastbana laboration }
\author{Mille Tedebring Thysell }
\date{\today}

\begin{document}

    \begin{titlepage}
        \begin{center}
            \vspace*{1cm}

            \Huge
            \textbf{Kastbana laboration}

            \vspace{0.5cm}
            \LARGE
            Kastbana med kulkanon

            \vspace{1.5cm}

            \textbf{Mille Tedebring Thysell}

            \vfill

            En laboration om kastbanor  \\
            Fysik 2

            \vspace{0.8cm}

            \includegraphics[width=0.4\textwidth]{../images/NTI Gymnasiet_Symbol_print_svart.png}

            \Large
            Teknikprogrammet\\
            NTI Gymnasiet\\
            Umeå\\
            \today

        \end{center}
    \end{titlepage}
% Om arbetet är långt har det en innehållsförteckning, annars kan den utelämnas
    \newpage
    \section{Material}
    \begin{enumerate}
        \item Kulkanon
        \item Kulor
        \item Linjal
        \item Kulmål
    \end{enumerate}

    \section{Metod}
    Under alla tester så användes tredje läget på kulkanonen.
        Det första som gjordes under laborationen var att testa skjuta med kulkanonen rakt upp för att mäta hur högt den
    skjöt. Det gjordes genom att vinkla kulkanonen 90 grader eller så att den riktades rakt upp, sedan hölls ett måttband
    vinkelrät med bordet uppåt till en rimlig mängd. När kanonen skjuts så mäts hur högt den for. Sedan går det att räkna
    ut utgångshastigheten med den datan.\\
    Med utgångshastigheten går det att räkna ut vart kulan kommer landa ifall den skjuts med samma kraft med vinkeln
    60 grader. \\



    \section{Resultat}





    \printbibliography

\end{document}
